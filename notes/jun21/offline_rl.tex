

\documentclass{article}
\title{Notes for: Offline Reinforcement Learning: Tutorial, Review, and Perspectives on Open Problems}
\author{M. Nef. \thanks{paper: https://arxiv.org/pdf/1812.10613.pdf}}
\date{\today}
\usepackage{amsmath}

\begin{document}
\maketitle

\section{Types of Reinforcement Learning}
\paragraph{Online Reinforcement Learning}
Reinforcement learning approach whereby an agent acts into the environment and directly updates the policy based solely on actions from said policy. Trajectories are also trained on in the same order they are collected.

The policy, \(\pi_k\), is updated with streaming data collected by \(\pi_k\) iteself.

There is no buffer in online reinforcement learning, unlike Off-Policy Reinforcement Learning.

\paragraph{Off-Policy Reinforcement Learning} The agent's experience is appended to a data buffer (replay buffer), \(\mathcal{D}\), and each new policy, \(\pi_k\) adds additional data to \(\mathcal{D}\).
\(\mathcal{D}\) will therefore comprise samples from \( \pi_0 , \pi_1 , \dots, \pi_k\) and this data will be used to train a new policy \(\pi_{k+1}\).

\paragraph{Offline Reinforcement Learning}
Reinforcement learning algorithms that utilize previously collected data, without additional online data collection.

Offline RL has some dataset, \(\mathcal{D}\), collected by potentially unknown policies (or policy) \(\pi_\beta\).
The training process doesn't interact with the MDP at all.

\section{General RL}
\pagebreak
\paragraph{Policy Gradients}

Since the following is the RL objective (what we want to maximise):

\begin{equation}
\label{eq:1}
J(\pi) = \mathbb{E}_{\tau \sim p_\pi (\tau)} \left[ \sum_{t=0}^H \gamma^t r(s_t, a_t) \right]
\end{equation}

where: \(r(s_t, a_t)\) is the reward function of the MDP. \(\gamma^t r(s_t , a_t)\) is the discounted reward. \(\ \sum_{t=0}^H \gamma^t r(s_t, a_t) \) is the return conditioned on some selected trajectory. \(\tau \sim p_\pi(\tau)\)
\(\tau\) is a random variable which samples from the Trajectory Distribution. \(\mathbb{E}\) We're looking at an expectation, so we're interested in the discounted future reward of each trajectory weighted by the probability of it occuring.


\paragraph{Approximate Dynamic Programming}

\paragraph{Actor-Critic Algorithms}

\paragraph{Model-Based Reinforcement Learning}

\end{document}
