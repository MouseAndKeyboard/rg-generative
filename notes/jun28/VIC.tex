

\documentclass{article}
\title{Notes for: Variational Intrinsic Control}
\author{M. Nef. \thanks{paper: https://arxiv.org/pdf/1812.10613.pdf}}
\date{\today}
\usepackage{amsmath}
\usepackage{amsfonts}
\usepackage[margin=1in]{geometry}

\DeclareMathOperator*{\argmax}{arg\,max}
\DeclareMathOperator*{\argmin}{arg\,min}

\begin{document}
\maketitle

\section{Preliminaries}

\subsection{What are ``options''}

An option is a procedure which can be executed by an agent. Once an option has been executed it will take actions for the agent and eventually, at some point, return control back to the agent.


The way an option is described symbolically doesn't pair perfectly with this description. 


\end{document}
